%%%%%%%%%%%%%%%%%%%%%%%%%%%%%%%%%%%%%%%%%
% Wenneker Assignment
% LaTeX Template
% Version 2.0 (12/1/2019)
%
% This template originates from:
% http://www.LaTeXTemplates.com
%
% Authors:
% Vel (vel@LaTeXTemplates.com)
% Frits Wenneker
%
% License:
% CC BY-NC-SA 3.0 (http://creativecommons.org/licenses/by-nc-sa/3.0/)
% 
%%%%%%%%%%%%%%%%%%%%%%%%%%%%%%%%%%%%%%%%%

%----------------------------------------------------------------------------------------
%	PACKAGES AND OTHER DOCUMENT CONFIGURATIONS
%----------------------------------------------------------------------------------------

\documentclass[11pt]{scrartcl} % Font size

%%%%%%%%%%%%%%%%%%%%%%%%%%%%%%%%%%%%%%%%%
% Wenneker Assignment
% Structure Specification File
% Version 2.0 (12/1/2019)
%
% This template originates from:
% http://www.LaTeXTemplates.com
%
% Authors:
% Vel (vel@LaTeXTemplates.com)
% Frits Wenneker
%
% License:
% CC BY-NC-SA 3.0 (http://creativecommons.org/licenses/by-nc-sa/3.0/)
% 
%%%%%%%%%%%%%%%%%%%%%%%%%%%%%%%%%%%%%%%%%

%----------------------------------------------------------------------------------------
%	PACKAGES AND OTHER DOCUMENT CONFIGURATIONS
%----------------------------------------------------------------------------------------

\usepackage{amsmath, amsfonts, amsthm} % Math packages

\usepackage{listings} % Code listings, with syntax highlighting

\usepackage[main = greek, english]{babel} % English language hyphenation

\usepackage{graphicx} % Required for inserting images
\graphicspath{{Figures/}{./}} % Specifies where to look for included images (trailing slash required)

\usepackage{booktabs} % Required for better horizontal rules in tables

\usepackage{dirtytalk} % Required for quoting.

\usepackage{float} % Added for hard placement of images.
\usepackage{subfig}

\usepackage[dvipsnames]{xcolor} % Added for extra colors.

\usepackage{tikz} % For colored boxes and more.

\numberwithin{equation}{section} % Number equations within sections (i.e. 1.1, 1.2, 2.1, 2.2 instead of 1, 2, 3, 4)
\numberwithin{figure}{section} % Number figures within sections (i.e. 1.1, 1.2, 2.1, 2.2 instead of 1, 2, 3, 4)
\numberwithin{table}{section} % Number tables within sections (i.e. 1.1, 1.2, 2.1, 2.2 instead of 1, 2, 3, 4)

\usepackage{enumitem} % Required for list customisation
\setlist{noitemsep} % No spacing between list items

\usepackage{minted} % Required for insertion of code
\usepackage{hyperref} % Include the hyperref package
\usepackage{url} % Include the url package for formatting URLs in bibliography

%----------------------------------------------------------------------------------------
%	DOCUMENT MARGINS
%----------------------------------------------------------------------------------------

\usepackage{geometry} % Required for adjusting page dimensions and margins

\geometry{
	paper=a4paper, % Paper size, change to letterpaper for US letter size
	top=1.5cm, % Top margin
	bottom=1.5cm, % Bottom margin
	left=3cm, % Left margin
	right=3cm, % Right margin
	headheight=0.5cm, % Header height
	footskip=0.7cm, % Space from the bottom margin to the baseline of the footer
	headsep=0.5cm, % Space from the top margin to the baseline of the header
	% showframe, % Uncomment to show how the type block is set on the page
}

%----------------------------------------------------------------------------------------
%	FONTS
%----------------------------------------------------------------------------------------

\usepackage[utf8]{inputenc} % Required for inputting international characters
\usepackage[T1]{fontenc} % Use 8-bit encoding

\usepackage{fourier} % Use the Adobe Utopia font for the document

%----------------------------------------------------------------------------------------
%	SECTION TITLES
%----------------------------------------------------------------------------------------

\usepackage{sectsty} % Allows customising section commands

\sectionfont{\vspace{6pt}\centering\normalfont\scshape} % \section{} styling
\subsectionfont{\normalfont\bfseries} % \subsection{} styling
\subsubsectionfont{\normalfont\itshape} % \subsubsection{} styling
\paragraphfont{\normalfont\scshape} % \paragraph{} styling

%----------------------------------------------------------------------------------------
%	HEADERS AND FOOTERS
%----------------------------------------------------------------------------------------

\usepackage{scrlayer-scrpage} % Required for customising headers and footers

\ohead*{} % Right header
\ihead*{} % Left header
\chead*{} % Centre header

\ofoot*{} % Right footer
\ifoot*{} % Left footer
\cfoot*{\pagemark} % Centre footer

\newcommand{\img}[3][htbp]{%
    \begin{figure}[h]
        \centering
        \fcolorbox{black}{white}{\includegraphics[height=#3em]{#2}}
        \ifx&#1&%
        \else
            \caption{#1}
        \fi
    \end{figure}
}

% Helper Macros

\newcommand{\en}[1]{\foreignlanguage{english}{#1}}

% Define a new command for including two images side by side
\newcommand{\twoimgs}[4]{%
    \begin{figure}[htbp]
        \centering
        \begin{minipage}[b]{0.4\textwidth}
            \fcolorbox{black}{white}{\includegraphics[width=\textwidth]{#1}}
            \caption{#2}
        \end{minipage}
        \hfill
        \begin{minipage}[b]{0.4\textwidth}
            \fcolorbox{black}{white}{\includegraphics[width=\textwidth]{#3}}
            \caption{#4}
        \end{minipage}
    \end{figure}
}
% Extra Formatting

\setlength{\parindent}{0em}
\setlength{\parskip}{0em}
 % Include the file specifying the document structure and custom commands
\usepackage{caption}
\usepackage{amsmath} % for the bmatrix environment

%----------------------------------------------------------------------------------------
%	TITLE SECTION
%----------------------------------------------------------------------------------------

\title{	
	\normalfont\normalsize
	\textsc{Πανεπιστήμιο Πατρών, Τμήμα Ηλεκτρολόγων Μηχανικών και Τεχνολογίας Υπολογιστών}\\ % Your university, school and/or department name(s)
	\vspace{25pt} % Whitespace
	\rule{\linewidth}{0.5pt}\\ % Thin top horizontal rule
	\vspace{20pt} % Whitespace
	{\huge \en{ART GALLERY}}\\ % The assignment title
	\vspace{12pt} % Whitespace
	\rule{\linewidth}{2pt}\\ % Thick bottom horizontal rule
	\vspace{12pt} % Whitespace
}

\author{\LARGE Ηλίας Ουζούνης \\ \en{up1083749}} % Your name

\date{\normalsize\today} % Today's date (\today) or a custom date

\begin{document}
\maketitle

\newpage

\section*{\en{General Approach}}
	Για αυτό το πρότζεκτ, υλοποιήθηκε μια πολύ \en{object oriented} προσέγγιση. Όλα τα αντικείμενα στην σκηνή (πίνακες, καρέκλες,
ακόμα και τοίχοι - πατώματα) κάνουν \en{inherit} από την κλάσση \en{Object} η οποία χειρίζεται όλα τα \en{rendering, textures, uniforms}
κλπ. Αυτό έδινε μια ευελιξία στις ιδιόητες των αντικειμένων όσο άλλαζε το πρότζεκτ από ερώτημα σε ερώτημα. 
Επιπλέον ήταν ένας τρόπος να διατηρηθεί ο κώδικας καθαρός και οργανωμένος καθώς το μέγεθος των αρχείων αυξανόταν συνεχώς.\\

Εκτός από τα αντικείμενα, όλα τα \en{FBOs} κληρονομούν από μία κλάση \en{FBO}. Έτσι όλα τα \en{FBOs} που χρειάστηκαν είχαν παρόμοια
μορφή και ήταν πιο εύκολη η εναλλαγή μεταξύ τους.\\

\section{\en{Cylindrical Room}}

Ένα τέλεια κυλινδικό δωμάτιο είναι αδύνατον να υλοποιηθεί στο \en{OpenGL} καθώς αυτό απαιτεί απεριόριστο αριθμό \en{vertices}.
Για να προσεγγιστεί ένας κύλινδρος, επιλέγουμε πολλά σημεία ισοκατανεμημένα στο περίγυρο ενός κύκλου και τα ενώνουμε, δημιουργώντας ένα πολύγωνο
με πολλές πλευρές. 
\img[\en{Wireframe} του κυλινδικού δωματίου]{images/wireframe.png}{25}
\\
Για το ταβάνι και το πάτωμα, στην αρχή είχα φτιάξει πολλά τρίγωνα από τις κορυφές που ορίστηκαν από τον κύκλο και το κέντρο του.
Αυτό δημιουργούσε πολλά λεπτά τρίγωνα που συνέκλιναν στο κέντρο του δωματίου. Αυτή η προσέγγιση ήταν πιο δύσκολη στα \en{UVs} για
τα \en{textures} του ταβανιού και του πατώματος. Αντί αυτού, επέλεξα να φτιάξω ένα μεγάλο παραλληλόγραμμο που καλύπτει το δωμάτιο.
Παρόλο που θα προεξέχει, αυτό δεν θα φαίνεται καθώς θα είναι κρυμμένο από τους τοίχους. Αυτό έκανε τα \en{UVs} πολύ πιο εύκολα και
τα \en{textures} πιο ρεαλιστικά.
\img[Ταβάνι και πάτωμα. Είναι σκιασμένα γιατί το φως μπλοκάρεται από τους τοίχους]{images/ceilandfloor.png}{25}
\\
Οι πίνακες τώρα είναι ισομοιρασμένοι στον κύκλο και έχουν τοποθετηθεί κοντά στον τοίχο. Επέλεξα να έχω 6 τελικά, 1 για κάθε
διαφορετικό \en{shading style} και έναν \en{default}. Ενδιαφέρουσα είναι η εναλλαγή δωματίου 'μπαίνοντας' μέσα στον κάθε πίνακα.
Αυτό επιτυγχάνεται θεωρώντας ένα \en{bounding box} για τον παίχτη και ελέγχοντας πότε συγκρούεται με τον πίνακα.
Το \en{bounding box} του παίχτη είναι ένα παραλληλεπίπεδο που ορίζεται με 8 σημεία και δεν περιστρέφεται μαζί με τον παίχτη, απλά
μετακινείται. Για να ελέγξουμε πότε αυτό το παραλληλεπίπεδο συγκρούεται με τον πίνακα αρχικά ελέγχουμε αν το \en{y} που βρίσκεται
ο παίχτης είναι κατάλληλο. Έπειτα βρίσκουμε το επίπεδο που ο κάθε πίνακας ορίζει από
το \en{normal} του. Τότε, παίρνουμε την εξίσωση του επιπέδου $ dot(normal, point) + d = 0 $ και βάζουμε τις κορυφές του 
παραλληλεπίπεδου. Αν για κάποιες κορυφές το $ dot(normal, point) + d $ είναι θετικό ενώ για κάποιο άλλο σημείο αρητικό, σημαίνει
ότι 2 σημεία του \en{bounding box} βρίσκονται σε αντίθετα μέρη του επιπέδου, άρα τέμνει το επίπεδο του πίνακα. Αν όλα είναι ομόσημα
σημαίνει πως βρίσκεται αποκλιστηκά στην μία πλευρά του επιπέδου και σίγουρα δεν ακουμπάει τον πίνακα.
\img[Τα επίπεδα που ορίζουν τα \en{normals} των πινάκων]{images/top_view_normals.png}{25}
\\\\
Αυτή η τεχνική δουλεύει αρκετά καλά από μόνη της γιατί τα επίπεδα βρίσκοντα εντός του δωματίου μόνο εκεί που θα ήταν οι πίνακες.
Έτσι δεν υπάρχει περίπτωση να τέμνει ο παίχτης το επίπεδο κάπου πολύ μακριά του πίνακα και να μεταφέρεται στο άλλο δωμάτιο αναπάντεχα.
Όμως, κοντά στα όρια του πίνακα έχουμε εσφαλμένες συγκρούσεις. Για να καλύψουμε και αυτή την περίπτωση προσθέτουμε μία ακόμα συνθήκη.
Περιστρέφοντας το \en{normal} του πίνακα κατά +90 και -90 μοίρες, παίρνουμε δύο κάθετα επίπεδα στον πίνακα. Για να θεωρηθεί η σύγκρουση
σωστή πρέπει το \en{position} του παίχτη να βρίσκεται στην πράσινη περιοχή, δηλαδή να δίνει θετικές τιμές αν το βάλουμε στην εξίσωση
των 2 επιπέδων.
\img[Τα δύο κάθετα επίπεδα και η έγκυρη περιοχή]{images/side_normals.png}{25}
\\\\
Όταν γίνεται τελικά η σύκρουση, ο παίχτης μεταφέρεται στο κατάλληλο δωμάτιο και του δίνεται μια αρχική ταχύτητα,
σαν να πηδάει έξω από τον πίνακα.
\clearpage

\section{Αντικείμενα και σκιές}
Κάθε στυλ \en{shading} αντιστοιχεί σε ένα κυβικό δωμάτιο. Απεικονίζουν μια \en{abstract} ερμηνεία του γνωστού πίνακα του
\en{Vincent Van Gogh, Bedroom in Arles}.
\twoimgs{images/bedroom.jpg}{\en{Bedroom in Arles}}{images/normal.png}{\en{Bedroom in OpenGL}}
\\\\
Για την σκίαση των αντικειμένων χρειαζόταν να φύγω από το \en{orthographic projection shading} που κάναμε στο εργαστήριο καθώς δεν έδινε
τα επιθυμητά αποτελέσματα. Οι πίνακες θα άφηναν σκιά μόνο στο πάτωμα παρόλο που η πηγή είναι μία λάμπα στο ταβάνι. Στην αρχή
δοκίμασα να αλλάξω το \en{orthographic projection} σε \en{perpective projection}. Αυτό άφηνε πολλά \en{artifacts} και δεν ήταν η σωστή
προσέγγιση.
\img[\en{Artifacts} από το \en{perpective projection}]{images/artifacts.png}{25}
\\
Η σωστή προσέγγιση ήταν η χρήση \en{point shadows} \cite{PointShadows}. Σε αυτή τη μέθοδο αντί για 1 \en{depth map} χρησιμοποιούμε 6, μία για κάθε
πλευρά ενός κύβου όπου γίνεται η προβολή. Χρειάζεται αλλαγή στον \en{depth shader} για να προστεθεί και ένας \en{geometry shader} που κάνει αυτή την
μετάφραση σε 6 \en{depth maps}. Για να βρούμε την σκίαση κάθε σημείου, παίρνουμε την προβολή του πάνω σε αυτόν τον κύβο.
\img[\en{Shadow Mapping Example}]{images/point_shadows.png}{20}

\section{\en{FBOs}}
Η εφαρμογή των διαφορετικών \en{artstyles} θα γίνεται με \en{post processing}. Τότε, χρειάζεται να αποθηκεύεται το κάθε \en{rendered frame}
σε ένα \en{texture} όπου θα γίνεται τυχόν επεξεργασία και έπειτα θα εμφανίζεται στην οθόνη. Για αυτόν τον σκοπό χρειαζόμαστε ένα \en{sceneFBO}
όπου έχει στο \en{color attachement} ένα \en{texture} όπου αποθηκεύεται το \en{rendered frame}. Έπειτα, αυτό το \en{texture} δίνεται
στον \en{default framebuffer} και γίνεται \en{rendered} πάνω σε ένα \en{plane} που καλύπτει όλη την οθόνη. Αυτό το \en{texture} μπορεί
να υποστεί επεξεργασία στην \en{CPU} πριν γίνει \en{rendered} ή καθώς γίνεται \en{rendered} στους \en{shaders}. Για κάθε \en{artstyle}
υπάρχει και ένας \en{shader} που του αντιστοιχεί και χρησιμοποιείται για το \en{rendering} του \en{texture} στο \en{plane} ενώ όπου
χρειάζεται υπάρχουν και συναρτήσεις \en{c++} για προεπεξεργασία.

\clearpage
\section{\en{Artstyles}}
\subsection{\en{Floyd-Steinberg Dithering}}
Το \en{Floyd-Steinberg Dithering} είναι μία μέθοδος για να μειώσουμε τον αριθμό των χρωμάτων σε μία εικόνα, χωρίς να χάσουμε πολύ
ποιότητα. Ο αλγόριθμος αυτός \cite{Floyd-Steinberg}, για κάθε \en{pixel} της εικόνας προσθέτει το \en{error} που προκύπτει από την απόκλιση του χρώματος
των προηγούμενων \en{pixels} και έπειτα κβαντίζει το χρώμα στην πλησιέστερη τιμή. Με αυτόν τον τρόπο, οι αποκλίσεις του κάθε \en{pixel}
μοιράζονται σε όλον τον πίνακα και τα σφάλματα κβάντισης είναι δύσκολο να παρατηρηθούν, ειδικά από απόσταση. Δεν χρειάζεται κάποια
περεταίρω επεξεργασία οπότε γίνεται \en{rendered} στο \en{plane} με τον \en{default shader}.
\img[\en{Floyd-Steinberg Dithering} με 3 χρώματα ανά κανάλι (27 σύνολο)]{images/floyd_steinberg.png}{25}
\\\\
Επειδή όμως ο αλγόριθμος είναι σειριακός, αφού χρειάζεται τα προηγούμενα \en{pixels} για να λειτουργήσει, δεν είναι κατάλληλος για
να γίνει στους \en{shaders}. Για αυτό, φορτώνεται το \en{texture} στην \en{CPU} και γίνεται η επεξεργασία εκεί. Επειδή όμως η
διαδικασία είναι πολύ αργή, χρειάστηκε πολλές βελτιστοποιήσεις για να είναι ικανοποιητική η ταχύτητα. Παραδείγματα είναι η
εκτενή χρήση \en{bitwise operations} όπου ήταν δυνατή και η κατάλληλη μέθοδος φόρτωσης-εκφόρτωσης του \en{texture}.
(Φόρτωση σε \en{GL\_BGRA} αντί για \en{GL\_RGB} \cite{TextureOptimizations}).

\subsection{\en{Painterly Rendering}}
Το \en{Painterly Rendering} είναι μια προσπάθεια να αποδοθεί μια εικόνα με τον τρόπο που θα την ζωγράφιζε ένας ζωγράφος.
Συνδιάζει τα ερωτήματα \en{4.b \& 4.c} καθώς μοιάζουν μεταξύ τους σε αποτέλεσμα. Αυτή είναι μία ακόμα τεχνική που απαιτεί
επεξεργασία στην \en{CPU}. Φορτώνουμε το \en{texture} στην \en{CPU} και εκτελούμε τον αλγόριθμο. \cite{Painterly}
Ο αλγόριθμός εφαρμόζεται σε πολλά στάδια με πολλά \en{iterations}. Σε κάθε \en{iteration} επιλέγεται ένα τυχαίο βήμα 
για το \en{x} και για το \en{y} και δημιουργούμε ένα \en{grid} από επιλεγμένα \en{pixels}. Σε αυτά δίνουμε και ένα τυχαίο
\en{offset} για να μην μοιάζει τόσο ομοιόμορφο. \en{Seed} για τις τυχαίες τιμές είναι οι τιμές χρωμάτων κάποιων \en{pixels}
ώστε να είναι ίδιο το αποτέλεσμα όσο είναι στάσιμος ο παίχτης αλλά να αλλάζει όσο κουνιέται.
Έχουμε επιλέξει τώρα κάποια \en{pixels} τα οποία θα λειτουργήσουν ως η βάση
του \en{Painterly rendering}. Σκοπός μας είναι να μοιράσουμε το χρώμα τους και στα γειτονικά \en{pixels} ώστε να φαίνεται
σαν να έχουν χρωματιστεί με μία βούρτσα, όλα μαζί. Ανάλογα με την διαφορά χρώματος των γειτονικών \en{pixels} επιλέγουμε και το
μέγεθος της βούρτσας. Αυτό γιατί ένας ζωγράφος θα ήταν πιο προσεκτικός σε λεπτομέρειες όπως στις άκρες των αντικειμένων όπου
υπάρχουν μεγάλες εναλλαγές. Χρωματίζουμε τώρα τα \en{pixels} γύρω από το επιλεγμένο με το χρώμα του. Δεν χρειάζεται κάποια
περεταίρω επεξεργασία οπότε γίνεται \en{rendered} στο \en{plane} με τον \en{default shader}.
\img[\en{Painterly Rendering} με 15 \en{iterations}]{images/painterly.png}{25}
\\\\
Αυτή η τεχνική είναι ικανοποιητική αλλά στις ακμές των αντικειμένων φαίνεται ότι ξεφεύγει. Ένας πραγματικός ζωγράφος θα ήταν ακόμα
πιο προσεκτικός σε εκείνες τις περιοχές. Δοκίμασα, με ένα \en{edge detection} να βρω αυτές τις ακμές και να χωρίσω τον πίνακα σε περιοχές
με έναν \en{dfs} αλγόριθμο. Ο \en{Painterly} αλγόριθμος θα επιτρεπόταν να αλλάξει το χρώμα μόνο σε \en{pixels} που βρίσκονται 
στην ίδια περιοχή σαν έναν ζωγράφο που ζωγραφίζει εντός των γραμμών. Αυτή η προσέγγιση έδινε λίγο καλύτερα αποτελέσματα 
αλλά με μεγάλο υπολογιστικό κόστος που δεν άξιζε τον κόπο.
\clearpage

\subsection{\en{Fish Eye Effect}}
Το \en{fish eye effect} είναι ουσιαστικά μια διαστρεύλωση των \en{UV values} των \en{pixels} στην εικόνα. Θέλουμε τα \en{pixels} στο
κέντρο να μοιάζουν μεγαλύτερα από αυτά στα άκρα ώστε να πετύχουμε το εφέ ενός κυρτού φακού, όπως τα μάτια ενός ψαριού.
Προσθέτουμε ένα \en{offset} στις \en{UV} του κάθε \en{fragment} ανάλογα με μια δύναμη της απόστασης του από το κέντρο. Έτσι τα
μακρινά \en{pixels} θα διαστρεβλωθούν πολύ ενώ τα κεντρικά λίγο. Επίσης αν ένα \en{pixel} είναι πολύ μακριά από το κέντρο,
χρωματίζεται μαύρο γιατί θα έπαιρνε τιμές εκτός του (0, 1) με την διαστρεύλωση.
\img[\en{Fish Eye Effect}]{images/fish_eye.png}{25}

\subsection{\en{Toon Shading}}
Το \en{Toon Shading} είναι μια τεχνική σκίασης που μοιάζει με κόμικς. Αντί για ομαλές μεταβάσεις χρωμάτων, έχουμε έντονες αλλαγές
μεταξύ \en{grayscale} τιμών. Επίσης, δίνεται έμφαση στα περιγράμματα βάφοντας τα με μαύρο που δίνει την εντύπωση ότι είναι
σχεδιασμένα με μολύβι. Για το \en{edge detection} χρησιμοποιήθηκαν τα \en{Sobel filters} \cite{Sobel}

\begin{align*}
	\text{\en{sx}} &= \begin{bmatrix}
		1.0 & 2.0 & 1.0 \\
		0.0 & 0.0 & 0.0 \\
	   -1.0 & -2.0 & -1.0
	\end{bmatrix}
	&
	\text{\en{sy}} &= \begin{bmatrix}
		1.0 & 0.0 & -1.0 \\
		2.0 & 0.0 & -2.0 \\
		1.0 & 0.0 & -1.0
	\end{bmatrix}
\end{align*}

\img[\en{Toon Shading}]{images/toon.png}{25}
Το \en{Gradient Magnitude} το περνάμε μετά από μία \en{smoothscale function} για να μειωθεί ο θόρυβος. Κάθε \en{fragment} που
είναι πάνω από το \en{threshold} για \en{edge} χρωματίζεται μαύρο. Τα υπόλοιπα κβαντίζονται σε 6 \en{grayscale} τιμές.

\subsection{\en{Chromatic Aberration}}
Με το \en{Chromatic Aberration} αποδίδεται ένα \en{trippy} εφέ στην εικόνα. Τα αντικείμενα φαίνονται διπλά και μοιάζουν να κουνιούνται
αλλά και να είναι στάσιμα ταυτόχρονα. Αυτό επιτυγχάνεται παίρνοντας διαφορετικά \en{UV values} για κάθε κανάλι χρώματος.
Το κάθε κανάλι αντιστοιχείται σε μία περιοδική συνάρτηση με διαφορετική φάση και για αυτό φαίνονται τα αντικείμενα να κουνιούνται.
Το εφέ είναι καλύτερο και πιο ξεκάθαρο σε βίντεο.
\img[\en{Chromatic Aberration}]{images/aberration.png}{25}
\clearpage

\section{\en{Bump Rendering}}
Προς το παρόν, οι πίνακες είναι ένα μαύρο επίπεδο. Θέλουμε να είναι ένα \en{preview} του δωματίου που τους αντιστοιχεί.
Χρειαζόμαστε δηλαδή ένα \en{texture} που να απεικονίζει την σκηνή σε κάθε δωμάτιο. Πριν το \en{mainloop} τότε
κάνουμε ένα \en{render pass} για κάθε δωμάτιο και αποθηκεύουμε την σκηνή σε ένα \en{texture}. Για αυτό θα χρειαστούμε
ένα νέο \en{FBO} στο οποίο θα μπορούμε να αλλάζουμε το \en{color attachment texture} κάθε φορά. Το \en{texture} μπορούμε
να το επεξεργαστούμε κανονικά, όπως στο \en{mainloop} για να φαίενται και το εφέ του κάθε δωματίου. Έπειτα, αυτό το \en{texture} θα
αποθηκεύεται ως το \en{diffuse texture} κάθε πίνακα.
\img[\en{Preview of the deafult room}]{images/preview.png}{25}

Για να αποδώσουμε τις λεπτομέρειες των επιφανειών, μπορούμε να αποθηκεύσουμε και τα \en{normals} της σκηνής σε ένα 
\en{bump texture} και να χρησιμοποιούμε τα \en{normals} από εκεί στον \en{fragment shader} αντί για την παρεμβολή των
\en{normals} των \en{vertices} που μας περιορίζουν αρκετά. Χρειαζόμαστε έναν νέο \en{FBO} με μεταβλητό \en{color attachment texture}
στο οποίο αντί για το χρώμα της σκηνής θα αποθηκεύουμε τo \en{normal} του κάθε \en{fragment}. 
$$ normal.x -> color.r | normal.y -> color.g | normal.z -> color.b $$
Επειδή όμως τα \en{normals} παίρνουν τιμές (-1, +1) ενώ τα \en{rgb} τιμές παίρνουν (0, 1) κάνουμε \en{scale} το \en{normal}
κατά 0.5 και προσθέτουμε 0.5 ώστε να ταυτίζονται οι περιοχές τιμών. Όταν τα πάρουμε από το \en{texture} αντιστρέφουμε την διαδικασία.
\img[\en{Normal Map} της σκηνής αν το βάλουμε ως \en{diffuse texture}]{images/normal_preview.png}{25}
\\\\
Τα \en{normal maps} όμως απεικονίζουν τα \en{normals} του αντικειμένου μόνο σε ένα συγκεκριμένο \en{orientation}. Όταν το αντικείμενο
περιστραφεί, τα \en{normals} από το \en{normal map} δίνουν πολύ λανθασμένα αποτελέσματα.
Για την χρήση αυτών των \en{normals} χρειάζεται να μεταφέρουμε τους υπολογισμούς του \en{phong shading} από το 
\en{world space} στο \en{tangent space} ώστε να μην μας νοιάζει η περιστροφή του αντικειμένου \cite{Bump}. 
Για κάθε τρίγωνο κάθε αντικειμένου χρειάζεται να υπλογίσουμε ένα \en{TBN matrix} που ορίζεται από τα \en{normals} 
του και τις \en{UV} συνιστώσες του. Αυτό δημιουργεί ένα νέο σύστημα συντεταγμένων όπου θα δουλέφουμε.
\img[\en{Tangent space coordinates}]{images/tbn.png}{25}
\clearpage
Μεταφέρουμε τις μεταβλητές που χρειάζονται για το \en{Phong shading} στο νέο σύστημα συντεταγμένων πολλαπλασιάζοντας με
το \en{inverse transpose TBN matrix}. Χρειαζόμαστε την θέση του \en{fragment}, την θέση του φωτός και την θέση του παρατηρητή.
Έπειτα μπορούμε να κάνουμε τοςυ υπολογισμούς όπως συνήθως. Επειδή δεν είχαν όλα τα αντικείμενα \en{normal maps}, με ένα \en{flag}
αποφασίζει ο \en{shader} αν θα χρησιμοποιήσει το \en{normal map} ή τα \en{normals} του αντικειμένου.
\\\\
Οι κορνίζες των πινάκων όμως είχαν μια δυσκολία στον υπολογισμό του \en{TBN matrix}. Για ορισμένα τρίγωνα οι υπολογισμοί του
\en{tangent vector} και του \en{bitangent vector} έβγαιναν \en{nan} ή \en{inf} επειδή 2 κορυφές του είχαν τις ίδιες \en{UV} συντεταγμένες.
\img[\en{Nan} και \en{inf} τιμές στον υπολογισμό του \en{TBN matrix}]{images/frames.png}{25}
\\\\ Για αυτό χρειάστηκε να θεσω ένα ακόμα \en{flag} αν θα κάνει ο \en{shader} τους υπολογισμούς στο \en{world space} αντί για το
\en{tangent space}.
\\\\
Για τον πίνακα με το \en{fish eye effect} έπρεπε να διεστρευλώσω και το \en{normal map} ώστε να υπάρχει σωστή αντιστοίχηση με
το \en{diffuse texture}.

\clearpage
\section{\en{Parallax Mapping}}
Για μεγαλύτερες λεπτομέρειες στις επιφάνειες χρησιμοποιείται κάποια μέθοδος \en{displacement mapping}. Με το \en{parallax mapping}
\cite{Parallax} μπορούμε να προσομοιώσουμε γεωμετρικές λεπτομέρειες χωρίς όμως να αλλάζουμε το \en{mesh} του αντικειμένου.
\\\\
Αρχικά αποθηκεύουμε πληροφορίες για το βάθος του κάθε αντικειμένου της σκηνής σε ένα \en{displacement texture} κάνοντας άλλο
ένα \en{render pass} πριν το \en{mainloop} με ένα διαφορετικό \en{FBO} όπου στο \en{color attachement} αποθηκεύει την \en{z}
συνιστώσα του κάθε \en{fragment}. 
Έπειτα μπορούμε να χρησιμοποιήσουμε αυτό το \en{texture} για να δώσουμε στους πίνακες ένα αίσθημα βάθους που δεν υπάρχει.
\\
Η βασική αρχή του \en{parallax mapping} είναι η αλλαγή των \en{UV coordinates} ανάλογα με το βάθος του \en{texture} σε
εκείνη την περιοχή. Δίνουμε την ψευδαπάτη του βάθους υπολογίζοντας άμα το αντικείμενο ήταν όντως ανάγλυφο, ποιο σημείο
θα έβλεπε ο παρατηρητής.
\img[\en{Parallax Occlusion Mapping}]{images/parallax_diagram.png}{20}
Στο κάθε \en{fragment} βλέπουμε από το \en{displacement texture} το βάθος σε εκείνο το σημείο. Προεκτίνουμε το \en{view direction vector}
όσο είναι το βάθος ή φτάσουμε σε ένα \en{threshold}. Αυτό είναι το νέο \en{UV} του \en{fragment} που θα χρησιμοποιηθεί. Επαναλαμβάνουμε
αυτή την διαδικασία μέχρι το \en{view vector} να φτάσει κάτω από το \en{depth value} του \en{UV}. Τότε θεωρούμε \en{UV} για αυτό το 
\en{fragment} τον μέσο όρο των 2 τελευταίων \en{UVs} και το χρησιμοποιούμε για το \en{texture lookup}.
\\\\ 
Με αυτήν την μέθοδο πετυχαίνουμε το εφέ του βάθους. Κοιτώντας από διαφορετικές γωνίες βλέπουμε τα αντικείμενα να κουνιούνται και
πράγματα που δεν ήταν εμφανά στην αρχή γιατί τα μπλόκαρε κάτι άλλο να είναι ορατά. Σε βίντεο είναι πιο εμφανές το εφέ.
\twoimgs{images/parallax1.png}{\en{Looking from the left}}{images/parallax2.png}{\en{Looking from the right}}

\clearpage
\selectlanguage{english}
\bibliographystyle{unsrt} % Specify bibliography style
\bibliography{references} % Include your bibliography file (references.bib)

\end{document}
