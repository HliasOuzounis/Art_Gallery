%%%%%%%%%%%%%%%%%%%%%%%%%%%%%%%%%%%%%%%%%
% Wenneker Assignment
% LaTeX Template
% Version 2.0 (12/1/2019)
%
% This template originates from:
% http://www.LaTeXTemplates.com
%
% Authors:
% Vel (vel@LaTeXTemplates.com)
% Frits Wenneker
%
% License:
% CC BY-NC-SA 3.0 (http://creativecommons.org/licenses/by-nc-sa/3.0/)
% 
%%%%%%%%%%%%%%%%%%%%%%%%%%%%%%%%%%%%%%%%%

%----------------------------------------------------------------------------------------
%	PACKAGES AND OTHER DOCUMENT CONFIGURATIONS
%----------------------------------------------------------------------------------------

\documentclass[11pt]{scrartcl} % Font size

\include{structure.tex} % Include the file specifying the document structure and custom commands
\usepackage{caption}
\usepackage{subcaption}

%----------------------------------------------------------------------------------------
%	TITLE SECTION
%----------------------------------------------------------------------------------------

\title{	
	\normalfont\normalsize
	\textsc{Πανεπιστήμιο Πατρών, Τμήμα Ηλεκτρολόγων Μηχανικών και Τεχνολογίας Υπολογιστών}\\ % Your university, school and/or department name(s)
	\vspace{25pt} % Whitespace
	\rule{\linewidth}{0.5pt}\\ % Thin top horizontal rule
	\vspace{20pt} % Whitespace
	{\huge Γραφικά και Εικονική Πραγματικότητα: \en{Art Gallery}}\\ % The assignment title
	\vspace{12pt} % Whitespace
	\rule{\linewidth}{2pt}\\ % Thick bottom horizontal rule
	\vspace{12pt} % Whitespace
}

\author{\LARGE Ηλίας Ουζούνης \\ \en{up1083749}} % Your name

\date{\normalsize\today} % Today's date (\today) or a custom date

\begin{document}
\maketitle

\newpage

\section*{A.1}
Δημιούργησα μια κλάση \en{Room} που κληρωνομούν τα \en{MainRoom} και \en{SecondaryRoom}. Το \en{MainRoom} είναι ένα
κυλινδικό δωμάτιο ενώ τα \en{SecondaryRooms} είναι κυβικά. 

Δημιοτργούνται και 5 πίνακες και γίνονται κατάλληλα \en{rendered} αν ο παίχτης βρίσκεται στο σωστό δωμάτιο που φαίνεται από
το \en{gamestate}.

Οι εναλλαγές μεταξύ των δωματίων γίνονται προς το παρόν με το πληκτρολόγιο.
1-5 για τα \en{SecondaryRooms} και 0 για το \en{MainRoom}. Σκοπός είναι στο μέλλον, με \en{collision detection} να γίνεται όταν ο 
παίχτης "μπει" μέσα στον πίνακα.


\section*{A.2}
Για τον φωτισμό υπάρχει μία πηγή στο κέντρο κάθε δωματίου. Για φωτισμό χρησιμοποίησα \en{Phong}. Για τις σκιές χρησιμοποίησα 
\en{perspective projection} για την πηγή καθώς βγάζει περισσότερο νόημα. Δυστυχώς, χρειάζεται πολύ μεγάλο \en{FOV} για να σκιάζει
καλά όλους τους πίνακες οπότε τα αποτελέσματα δεν είναι τόσο καλά. Υπάρχει θέμα \en{quantization} (νομίζω).

Για τα αντικείμενα, κάθε δωμάτιο έχει έναν πίνακα με αντικείμενα τα οποία σε ένα \en{for loop} γίνονται \en{rendered} όταν γίνεται 
και το δωμάτιο.

\section*{A.3}
Έχω ένα πίνακα με 6 \en{shaders} και ανάλογα με το \en{gamestate} επιλέγω το κατάλληλο.

\end{document}
