%%%%%%%%%%%%%%%%%%%%%%%%%%%%%%%%%%%%%%%%%
% Wenneker Assignment
% LaTeX Template
% Version 2.0 (12/1/2019)
%
% This template originates from:
% http://www.LaTeXTemplates.com
%
% Authors:
% Vel (vel@LaTeXTemplates.com)
% Frits Wenneker
%
% License:
% CC BY-NC-SA 3.0 (http://creativecommons.org/licenses/by-nc-sa/3.0/)
% 
%%%%%%%%%%%%%%%%%%%%%%%%%%%%%%%%%%%%%%%%%

%----------------------------------------------------------------------------------------
%	PACKAGES AND OTHER DOCUMENT CONFIGURATIONS
%----------------------------------------------------------------------------------------

\documentclass[11pt]{scrartcl} % Font size

%%%%%%%%%%%%%%%%%%%%%%%%%%%%%%%%%%%%%%%%%
% Wenneker Assignment
% Structure Specification File
% Version 2.0 (12/1/2019)
%
% This template originates from:
% http://www.LaTeXTemplates.com
%
% Authors:
% Vel (vel@LaTeXTemplates.com)
% Frits Wenneker
%
% License:
% CC BY-NC-SA 3.0 (http://creativecommons.org/licenses/by-nc-sa/3.0/)
% 
%%%%%%%%%%%%%%%%%%%%%%%%%%%%%%%%%%%%%%%%%

%----------------------------------------------------------------------------------------
%	PACKAGES AND OTHER DOCUMENT CONFIGURATIONS
%----------------------------------------------------------------------------------------

\usepackage{amsmath, amsfonts, amsthm} % Math packages

\usepackage{listings} % Code listings, with syntax highlighting

\usepackage[main = greek, english]{babel} % English language hyphenation

\usepackage{graphicx} % Required for inserting images
\graphicspath{{Figures/}{./}} % Specifies where to look for included images (trailing slash required)

\usepackage{booktabs} % Required for better horizontal rules in tables

\usepackage{dirtytalk} % Required for quoting.

\usepackage{float} % Added for hard placement of images.
\usepackage{subfig}

\usepackage[dvipsnames]{xcolor} % Added for extra colors.

\usepackage{tikz} % For colored boxes and more.

\numberwithin{equation}{section} % Number equations within sections (i.e. 1.1, 1.2, 2.1, 2.2 instead of 1, 2, 3, 4)
\numberwithin{figure}{section} % Number figures within sections (i.e. 1.1, 1.2, 2.1, 2.2 instead of 1, 2, 3, 4)
\numberwithin{table}{section} % Number tables within sections (i.e. 1.1, 1.2, 2.1, 2.2 instead of 1, 2, 3, 4)

\usepackage{enumitem} % Required for list customisation
\setlist{noitemsep} % No spacing between list items

\usepackage{minted} % Required for insertion of code
\usepackage{hyperref} % Include the hyperref package
\usepackage{url} % Include the url package for formatting URLs in bibliography

%----------------------------------------------------------------------------------------
%	DOCUMENT MARGINS
%----------------------------------------------------------------------------------------

\usepackage{geometry} % Required for adjusting page dimensions and margins

\geometry{
	paper=a4paper, % Paper size, change to letterpaper for US letter size
	top=1.5cm, % Top margin
	bottom=1.5cm, % Bottom margin
	left=3cm, % Left margin
	right=3cm, % Right margin
	headheight=0.5cm, % Header height
	footskip=0.7cm, % Space from the bottom margin to the baseline of the footer
	headsep=0.5cm, % Space from the top margin to the baseline of the header
	% showframe, % Uncomment to show how the type block is set on the page
}

%----------------------------------------------------------------------------------------
%	FONTS
%----------------------------------------------------------------------------------------

\usepackage[utf8]{inputenc} % Required for inputting international characters
\usepackage[T1]{fontenc} % Use 8-bit encoding

\usepackage{fourier} % Use the Adobe Utopia font for the document

%----------------------------------------------------------------------------------------
%	SECTION TITLES
%----------------------------------------------------------------------------------------

\usepackage{sectsty} % Allows customising section commands

\sectionfont{\vspace{6pt}\centering\normalfont\scshape} % \section{} styling
\subsectionfont{\normalfont\bfseries} % \subsection{} styling
\subsubsectionfont{\normalfont\itshape} % \subsubsection{} styling
\paragraphfont{\normalfont\scshape} % \paragraph{} styling

%----------------------------------------------------------------------------------------
%	HEADERS AND FOOTERS
%----------------------------------------------------------------------------------------

\usepackage{scrlayer-scrpage} % Required for customising headers and footers

\ohead*{} % Right header
\ihead*{} % Left header
\chead*{} % Centre header

\ofoot*{} % Right footer
\ifoot*{} % Left footer
\cfoot*{\pagemark} % Centre footer

\newcommand{\img}[3][htbp]{%
    \begin{figure}[h]
        \centering
        \fcolorbox{black}{white}{\includegraphics[height=#3em]{#2}}
        \ifx&#1&%
        \else
            \caption{#1}
        \fi
    \end{figure}
}

% Helper Macros

\newcommand{\en}[1]{\foreignlanguage{english}{#1}}

% Define a new command for including two images side by side
\newcommand{\twoimgs}[4]{%
    \begin{figure}[htbp]
        \centering
        \begin{minipage}[b]{0.4\textwidth}
            \fcolorbox{black}{white}{\includegraphics[width=\textwidth]{#1}}
            \caption{#2}
        \end{minipage}
        \hfill
        \begin{minipage}[b]{0.4\textwidth}
            \fcolorbox{black}{white}{\includegraphics[width=\textwidth]{#3}}
            \caption{#4}
        \end{minipage}
    \end{figure}
}
% Extra Formatting

\setlength{\parindent}{0em}
\setlength{\parskip}{0em}
 % Include the file specifying the document structure and custom commands
\usepackage{caption}
\usepackage{subcaption}

%----------------------------------------------------------------------------------------
%	TITLE SECTION
%----------------------------------------------------------------------------------------

\title{	
	\normalfont\normalsize
	\textsc{Πανεπιστήμιο Πατρών, Τμήμα Ηλεκτρολόγων Μηχανικών και Τεχνολογίας Υπολογιστών}\\ % Your university, school and/or department name(s)
	\vspace{25pt} % Whitespace
	\rule{\linewidth}{0.5pt}\\ % Thin top horizontal rule
	\vspace{20pt} % Whitespace
	{\huge \en{ART GALLERY}}\\ % The assignment title
	\vspace{12pt} % Whitespace
	\rule{\linewidth}{2pt}\\ % Thick bottom horizontal rule
	\vspace{12pt} % Whitespace
}

\author{\LARGE Ηλίας Ουζούνης \\ \en{up1083749}} % Your name

\date{\normalsize\today} % Today's date (\today) or a custom date

\begin{document}
\maketitle

\newpage

\section*{\en{General Approach}}
	Για αυτό το πρότζεκτ, υλοποιήθηκε μια πολύ \en{object oriented} προσέγγιση. Όλα τα αντικείμενα στην σκηνή (πίνακες, καρέκλες,
ακόμα και τοίχοι - πατώματα) κάνουν \en{inherit} από την κλάσση \en{Object} η οποία χειρίζεται όλα τα \en{rendering, textures, uniforms}
κλπ. Αυτό έδινε μια ευελιξία στις ιδιόητες των αντικειμένων όσο άλλαζε το πρότζεκτ από ερώτημα σε ερώτημα. 
Επιπλέον ήταν ένας τρόπος να διατηρηθεί ο κώδικας καθαρός και οργανωμένος καθώς το μέγεθος των αρχείων αυξανόταν συνεχώς.\\

Εκτός από τα αντικείμενα, όλα τα \en{FBOs} κληρονομούν από μία κλάση \en{FBO}. Έτσι όλα τα \en{FBOs} που χρειάστηκαν είχαν παρόμοια
μορφή και ήταν πιο εύκολη η εναλλαγή μεταξύ τους.\\

\section{\en{Cylindrical Room}}

Ένα τέλεια κυλινδικό δωμάτιο είναι αδύνατον να υλοποιηθεί στο \en{OpenGL} καθώς αυτό απαιτεί απεριόριστο αριθμό \en{vertices}.
Για να προσεγγιστεί ένας κύλινδρος, επιλέγουμε πολλά σημεία ισοκατανεμημένα στο περίγυρο ενός κύκλου και τα ενώνουμε, δημιουργώντας ένα πολύγωνο
με πολλές πλευρές. 
\img[\en{Wireframe} του κυλινδικού δωματίου]{images/wireframe.png}{25}
\\
Για το ταβάνι και το πάτωμα, στην αρχή είχα φτιάξει πολλά τρίγωνα από τις κορυφές που ορίστηκαν από τον κύκλο και το κέντρο του.
Αυτό δημιουργούσε πολλά λεπτά τρίγωνα που συνέκλιναν στο κέντρο του δωματίου. Αυτή η προσέγγιση ήταν πιο δύσκολη στα \en{UVs} για
τα \en{textures} του ταβανιού και του πατώματος. Αντί αυτού, επέλεξα να φτιάξω ένα μεγάλο παραλληλόγραμμο που καλύπτει το δωμάτιο.
Παρόλο που θα προεξέχει, αυτό δεν θα φαίνεται καθώς θα είναι κρυμμένο από τους τοίχους. Αυτό έκανε τα \en{UVs} πολύ πιο εύκολα και
τα \en{textures} πιο ρεαλιστικά.
\img[Ταβάνι και πάτωμα. Είναι σκιασμένα γιατί το φως μπλοκάρεται από τους τοίχους]{images/ceilandfloor.png}{25}
\\
Οι πίνακες τώρα είναι ισομοιρασμένοι στον κύκλο και έχουν τοποθετηθεί κοντά στον τοίχο. Επέλεξα να έχω 6 τελικά, 1 για κάθε
διαφορετικό \en{shading style} και έναν \en{default}. Ενδιαφέρουσα είναι η εναλλαγή δωματίου 'μπαίνοντας' μέσα στον κάθε πίνακα.
Αυτό επιτυγχάνεται θεωρώντας ένα \en{bounding box} για τον παίχτη και ελέγχοντας πότε συγκρούεται με τον πίνακα.
Το \en{bounding box} του παίχτη είναι ένα παραλληλεπίπεδο που ορίζεται με 8 σημεία και δεν περιστρέφεται μαζί με τον παίχτη, απλά
μετακινείται. Για να ελέγξουμε πότε αυτό το παραλληλεπίπεδο συγκρούεται με τον πίνακα αρχικά ελέγχουμε αν το \en{y} που βρίσκεται
ο παίχτης είναι κατάλληλο. Έπειτα βρίσκουμε το επίπεδο που ο κάθε πίνακας ορίζει από
το \en{normal} του. Τότε, παίρνουμε την εξίσωση του επιπέδου $ dot(normal, point) + d = 0 $ και βάζουμε τις κορυφές του 
παραλληλεπίπεδου. Αν για κάποιες κορυφές το $ dot(normal, point) + d $ είναι θετικό ενώ για κάποιο άλλο σημείο αρητικό, σημαίνει
ότι 2 σημεία του \en{bounding box} βρίσκονται σε αντίθετα μέρη του επιπέδου, άρα τέμνει το επίπεδο του πίνακα. Αν όλα είναι ομόσημα
σημαίνει πως βρίσκεται αποκλιστηκά στην μία πλευρά του επιπέδου και σίγουρα δεν ακουμπάει τον πίνακα.
\img[Τα επίπεδα που ορίζουν τα \en{normals} των πινάκων]{images/top_view_normals.png}{25}
\\\\
Αυτή η τεχνική δουλεύει αρκετά καλά από μόνη της γιατί τα επίπεδα βρίσκοντα εντός του δωματίου μόνο εκεί που θα ήταν οι πίνακες.
Έτσι δεν υπάρχει περίπτωση να τέμνει ο παίχτης το επίπεδο κάπου πολύ μακριά του πίνακα και να μεταφέρεται στο άλλο δωμάτιο αναπάντεχα.
Όμως, κοντά στα όρια του πίνακα έχουμε εσφαλμένες συγκρούσεις. Για να καλύψουμε και αυτή την περίπτωση προσθέτουμε μία ακόμα συνθήκη.
Περιστρέφοντας το \en{normal} του πίνακα κατά +90 και -90 μοίρες, παίρνουμε δύο κάθετα επίπεδα στον πίνακα. Για να θεωρηθεί η σύγκρουση
σωστή πρέπει το \en{position} του παίχτη να βρίσκεται στην πράσινη περιοχή, δηλαδή να δίνει θετικές τιμές αν το βάλουμε στην εξίσωση
των 2 επιπέδων.
\img[Τα δύο κάθετα επίπεδα και η έγκυρη περιοχή]{images/side_normals.png}{25}
\\\\
Όταν γίνεται τελικά η σύκρουση, ο παίχτης μεταφέρεται στο κατάλληλο δωμάτιο και του δίνεται μια αρχική ταχύτητα,
σαν να πηδάει έξω από τον πίνακα.
\clearpage

\section{Αντικείμενα και σκιές}

\end{document}
